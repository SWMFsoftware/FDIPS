\documentclass[12pt]{article}

\title{FDIPS User Manual \\ \large Code Version 1.00}
\author{G\'abor T\'oth, Bart van der Holst, Zhenguang Huang}

\begin{document}

\maketitle

\section{Introduction}

The Finite Difference Iterative Potential field Solver (FDIPS) can be
used to obtain a potential field solution from a magnetogram.

\subsection{Quick Start}

\input ../../README

\subsection{Preparing Input File}

The magnetogram files are usually provided in FITS format. We provide
IDL scripts that can read these files and convert them into a plain
text files. You need IDL installed. 

First rename the FITSFILE as
\begin{verbatim}
  mv original_fits_file fitstile.fits
\end{verbatim}
Then start IDL in the same directory and run the {\tt fits\_to\_asci.pro}
script in the {\tt input} directory:
\begin{verbatim}
  IDL> input/fits_to_asci.pro
\end{verbatim}
This script will generate 3 files:
\begin{verbatim}
   fitsfile.H        -- header information
   fitsfile_idl.out  -- ASCII file containing longitude, latitude, Br
   fitsfile.dat      -- ASCII magnetogram file containing array size and Br
\end{verbatim}
Only the last file is needed for FDIPS. It can be renamed to any name.

Note that you can create input files by other means too. The format of the
acii magnetogram file is described at the \#MAGNETOGRAM command.

\subsection{Compiling FDIPS}

To create a serial executable type
\begin{verbatim}
  make FDIPS1
\end{verbatim}
For larger problems, the parallel code may be preferred. This requires
a working MPI library and a parallel (or at least multi-core) machine. Type
\begin{verbatim}
  make FDIPS
\end{verbatim}
in the main directory. The executables will be in the {\tt bin/} directory.

\subsection{Running FDIPS}

Create a run directory with
\begin{verbatim}
  make run
\end{verbatim}
and move the ASCII magnetogram file into it. Then edit the input parameter
file {\tt run/FDIPS.in} either with a standard editor, or by running the
parameter editor Graphical User Interface (GUI) with
\begin{verbatim}
  share/Scripts/ParamEditor.pl run/FDIPS.in
\end{verbatim}
You can also consult the next section or the PARAM.XML file to see
how the input parameters should be set.

Once the {\tt FDIPS.in} file is correctly setup, the serial version of FDIPS
can be run as
\begin{verbatim}
  ./FDIPS1.exe
\end{verbatim}
while the parallel version as
\begin{verbatim}
  mpirun -np 8 FDIPS.exe
\end{verbatim}
Note that the number of processors should match the data decomposition defined
in FDIPS.in (see the description of the \#PARALLEL command)

The serial code will create 1 or 2 complete output files that can be used
by another code or visualized. The parallel code will produce separate
files per processor. These need to be combined with the {\tt redistribute.pl}
script. For example if the domain is decomposed into $4 \times 2$ 
subdomains
in the $\theta, \phi$ coordinates, then the field and potential data
can be collected into complete files using
\begin{verbatim}
  ./redistribute.pl fdips_field_np010402_000.out fdips_field.out
  ./redistribute.pl fdips_pot_np010204_000.out fdips_pot.out
\end{verbatim}

\input ../../PARAM.xmltex

\end{document}

